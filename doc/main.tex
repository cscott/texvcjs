% !TeX spellcheck = en_US
\documentclass[a4paper,12pt]{article}
\usepackage{amsmath}
\usepackage{amsfonts}
\usepackage{amssymb}
\usepackage{bbold}
\usepackage{cancel}
\author{Moritz Schubotz}
\title{Technical details on texvc identifiere extraction}
\begin{document}
\maketitle
\section{Introduction}
This document describes which mathematical symbols are identified as identifiers.
In general every single Latin letter [a-zA-Z] is regarded as identifier.
In addition, we accept multi-letter-subscripts that match [0-9a-zA-Z]+, such as $a_0$ but also $\varepsilon_{ijk}$.
Moreover, the Literals described in section \ref{sc.lit}, and the Identifier variants (section \ref{sc.var}) are supported.
\section{Literals}\label{sc.lit}
The following literals are supported:

\texttt{\textbackslash Bbbk} is rendered as $\Bbbk$


\texttt{\textbackslash Delta} is rendered as $\Delta$


\texttt{\textbackslash Finv} is rendered as $\Finv$


\texttt{\textbackslash Game} is rendered as $\Game$


\texttt{\textbackslash Gamma} is rendered as $\Gamma$


\texttt{\textbackslash Lambda} is rendered as $\Lambda$


\texttt{\textbackslash Omega} is rendered as $\Omega$


\texttt{\textbackslash P} is rendered as $\P$


\texttt{\textbackslash Phi} is rendered as $\Phi$


\texttt{\textbackslash Pi} is rendered as $\Pi$


\texttt{\textbackslash Psi} is rendered as $\Psi$


\texttt{\textbackslash S} is rendered as $\S$


\texttt{\textbackslash Sigma} is rendered as $\Sigma$


\texttt{\textbackslash Theta} is rendered as $\Theta$


\texttt{\textbackslash Xi} is rendered as $\Xi$


\texttt{\textbackslash aleph} is rendered as $\aleph$


\texttt{\textbackslash alpha} is rendered as $\alpha$


\texttt{\textbackslash amalg} is rendered as $\amalg$


\texttt{\textbackslash backepsilon} is rendered as $\backepsilon$


\texttt{\textbackslash beta} is rendered as $\beta$


\texttt{\textbackslash beth} is rendered as $\beth$


\texttt{\textbackslash chi} is rendered as $\chi$


\texttt{\textbackslash complement} is rendered as $\complement$


\texttt{\textbackslash daleth} is rendered as $\daleth$


\texttt{\textbackslash delta} is rendered as $\delta$


\texttt{\textbackslash digamma} is rendered as $\digamma$


\texttt{\textbackslash ell} is rendered as $\ell$


\texttt{\textbackslash epsilon} is rendered as $\epsilon$


\texttt{\textbackslash eta} is rendered as $\eta$


\texttt{\textbackslash eth} is rendered as $\eth$


\texttt{\textbackslash flat} is rendered as $\flat$


\texttt{\textbackslash gamma} is rendered as $\gamma$


\texttt{\textbackslash gimel} is rendered as $\gimel$


\texttt{\textbackslash hslash} is rendered as $\hslash$


\texttt{\textbackslash imath} is rendered as $\imath$


\texttt{\textbackslash intercal} is rendered as $\intercal$


\texttt{\textbackslash iota} is rendered as $\iota$


\texttt{\textbackslash jmath} is rendered as $\jmath$


\texttt{\textbackslash kappa} is rendered as $\kappa$


\texttt{\textbackslash lambda} is rendered as $\lambda$


\texttt{\textbackslash mho} is rendered as $\mho$


\texttt{\textbackslash mu} is rendered as $\mu$


\texttt{\textbackslash natural} is rendered as $\natural$


\texttt{\textbackslash nu} is rendered as $\nu$


\texttt{\textbackslash omega} is rendered as $\omega$


\texttt{\textbackslash phi} is rendered as $\phi$


\texttt{\textbackslash pi} is rendered as $\pi$


\texttt{\textbackslash pitchfork} is rendered as $\pitchfork$


\texttt{\textbackslash psi} is rendered as $\psi$


\texttt{\textbackslash rho} is rendered as $\rho$


\texttt{\textbackslash sigma} is rendered as $\sigma$


\texttt{\textbackslash tau} is rendered as $\tau$


\texttt{\textbackslash theta} is rendered as $\theta$


\texttt{\textbackslash top} is rendered as $\top$


\texttt{\textbackslash varepsilon} is rendered as $\varepsilon$


\texttt{\textbackslash varkappa} is rendered as $\varkappa$


\texttt{\textbackslash varnothing} is rendered as $\varnothing$


\texttt{\textbackslash varphi} is rendered as $\varphi$


\texttt{\textbackslash varpi} is rendered as $\varpi$


\texttt{\textbackslash varrho} is rendered as $\varrho$


\texttt{\textbackslash varsigma} is rendered as $\varsigma$


\texttt{\textbackslash vartheta} is rendered as $\vartheta$


\texttt{\textbackslash wp} is rendered as $\wp$


\texttt{\textbackslash xi} is rendered as $\xi$


\texttt{\textbackslash zeta} is rendered as $\zeta$



\section{Identifier variants}\label{sc.var}
The following variants are supported\footnote{Note that \texttt{\textbackslash mathcal} is not available for lowercase Latin letters.}:

\texttt{\textbackslash Bbb} applied on $x,X$ is rendered as $\Bbb{x},\Bbb{X}$


\texttt{\textbackslash acute} applied on $x,X$ is rendered as $\acute{x},\acute{X}$


\texttt{\textbackslash bar} applied on $x,X$ is rendered as $\bar{x},\bar{X}$


\texttt{\textbackslash bcancel} applied on $x,X$ is rendered as $\bcancel{x},\bcancel{X}$


\texttt{\textbackslash bmod} applied on $x,X$ is rendered as $\bmod{x},\bmod{X}$


\texttt{\textbackslash bold} applied on $x,X$ is rendered as $\bold{x},\bold{X}$


\texttt{\textbackslash boldsymbol} applied on $x,X$ is rendered as $\boldsymbol{x},\boldsymbol{X}$


\texttt{\textbackslash breve} applied on $x,X$ is rendered as $\breve{x},\breve{X}$


\texttt{\textbackslash cancel} applied on $x,X$ is rendered as $\cancel{x},\cancel{X}$


\texttt{\textbackslash check} applied on $x,X$ is rendered as $\check{x},\check{X}$


\texttt{\textbackslash ddot} applied on $x,X$ is rendered as $\ddot{x},\ddot{X}$


\texttt{\textbackslash dot} applied on $x,X$ is rendered as $\dot{x},\dot{X}$


\texttt{\textbackslash emph} applied on $x,X$ is rendered as $\emph{x},\emph{X}$


\texttt{\textbackslash grave} applied on $x,X$ is rendered as $\grave{x},\grave{X}$


\texttt{\textbackslash hat} applied on $x,X$ is rendered as $\hat{x},\hat{X}$


\texttt{\textbackslash mathbb} applied on $x,X$ is rendered as $\mathbb{x},\mathbb{X}$


\texttt{\textbackslash mathbf} applied on $x,X$ is rendered as $\mathbf{x},\mathbf{X}$


\texttt{\textbackslash mathbin} applied on $x,X$ is rendered as $\mathbin{x},\mathbin{X}$


\texttt{\textbackslash mathcal} applied on $x,X$ is rendered as $\mathcal{x},\mathcal{X}$


\texttt{\textbackslash mathclose} applied on $x,X$ is rendered as $\mathclose{x},\mathclose{X}$


\texttt{\textbackslash mathfrak} applied on $x,X$ is rendered as $\mathfrak{x},\mathfrak{X}$


\texttt{\textbackslash mathit} applied on $x,X$ is rendered as $\mathit{x},\mathit{X}$


\texttt{\textbackslash mathop} applied on $x,X$ is rendered as $\mathop{x},\mathop{X}$


\texttt{\textbackslash mathopen} applied on $x,X$ is rendered as $\mathopen{x},\mathopen{X}$


\texttt{\textbackslash mathord} applied on $x,X$ is rendered as $\mathord{x},\mathord{X}$


\texttt{\textbackslash mathpunct} applied on $x,X$ is rendered as $\mathpunct{x},\mathpunct{X}$


\texttt{\textbackslash mathrel} applied on $x,X$ is rendered as $\mathrel{x},\mathrel{X}$


\texttt{\textbackslash mathrm} applied on $x,X$ is rendered as $\mathrm{x},\mathrm{X}$


\texttt{\textbackslash mathsf} applied on $x,X$ is rendered as $\mathsf{x},\mathsf{X}$


\texttt{\textbackslash mathtt} applied on $x,X$ is rendered as $\mathtt{x},\mathtt{X}$


\texttt{\textbackslash overleftarrow} applied on $x,X$ is rendered as $\overleftarrow{x},\overleftarrow{X}$


\texttt{\textbackslash overleftrightarrow} applied on $x,X$ is rendered as $\overleftrightarrow{x},\overleftrightarrow{X}$


\texttt{\textbackslash overline} applied on $x,X$ is rendered as $\overline{x},\overline{X}$


\texttt{\textbackslash overrightarrow} applied on $x,X$ is rendered as $\overrightarrow{x},\overrightarrow{X}$


\texttt{\textbackslash textbf} applied on $x,X$ is rendered as $\textbf{x},\textbf{X}$


\texttt{\textbackslash textit} applied on $x,X$ is rendered as $\textit{x},\textit{X}$


\texttt{\textbackslash textrm} applied on $x,X$ is rendered as $\textrm{x},\textrm{X}$


\texttt{\textbackslash textsf} applied on $x,X$ is rendered as $\textsf{x},\textsf{X}$


\texttt{\textbackslash texttt} applied on $x,X$ is rendered as $\texttt{x},\texttt{X}$


\texttt{\textbackslash tilde} applied on $x,X$ is rendered as $\tilde{x},\tilde{X}$


\texttt{\textbackslash underline} applied on $x,X$ is rendered as $\underline{x},\underline{X}$


\texttt{\textbackslash vec} applied on $x,X$ is rendered as $\vec{x},\vec{X}$


\texttt{\textbackslash widehat} applied on $x,X$ is rendered as $\widehat{x},\widehat{X}$


\texttt{\textbackslash widetilde} applied on $x,X$ is rendered as $\widetilde{x},\widetilde{X}$


\texttt{\textbackslash xcancel} applied on $x,X$ is rendered as $\xcancel{x},\xcancel{X}$


\texttt{\textbackslash xleftarrow} applied on $x,X$ is rendered as $\xleftarrow{x},\xleftarrow{X}$


\texttt{\textbackslash xrightarrow} applied on $x,X$ is rendered as $\xrightarrow{x},\xrightarrow{X}$


\end{document}
