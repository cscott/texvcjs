% !TeX spellcheck = en_US
\documentclass[a4paper,12pt]{article}
\usepackage{amsmath}
\usepackage{amsfonts}
\usepackage{amssymb}
\usepackage{bbold}
\usepackage{cancel}
\author{Moritz Schubotz}
\title{Technical details on texvc identifiere extraction}
\begin{document}
\maketitle
\section{Introduction}
This document describes which mathematical symbols are identified as identifiers.
In general every single Latin letter [a-zA-Z] is regarded as identifier.
In addition, we accept multi-letter-subscripts that match [0-9a-zA-Z]+, such as $a_0$ but also $\varepsilon_{ijk}$.
Moreover, the Literals described in section \ref{sc.lit}, and the Identifier variants (section \ref{sc.var}) are supported.
\section{Literals}\label{sc.lit}
The following literals are supported:

\texttt{\textbackslash aleph} is rendered as $\aleph$


\texttt{\textbackslash alpha} is rendered as $\alpha$


\texttt{\textbackslash amalg} is rendered as $\amalg$


\texttt{\textbackslash backepsilon} is rendered as $\backepsilon$


\texttt{\textbackslash Bbbk} is rendered as $\Bbbk$


\texttt{\textbackslash beta} is rendered as $\beta$


\texttt{\textbackslash beth} is rendered as $\beth$


\texttt{\textbackslash chi} is rendered as $\chi$


\texttt{\textbackslash complement} is rendered as $\complement$


\texttt{\textbackslash daleth} is rendered as $\daleth$


\texttt{\textbackslash delta} is rendered as $\delta$


\texttt{\textbackslash Delta} is rendered as $\Delta$


\texttt{\textbackslash digamma} is rendered as $\digamma$


\texttt{\textbackslash ell} is rendered as $\ell$


\texttt{\textbackslash epsilon} is rendered as $\epsilon$


\texttt{\textbackslash eta} is rendered as $\eta$


\texttt{\textbackslash eth} is rendered as $\eth$


\texttt{\textbackslash Finv} is rendered as $\Finv$


\texttt{\textbackslash flat} is rendered as $\flat$


\texttt{\textbackslash Game} is rendered as $\Game$


\texttt{\textbackslash gamma} is rendered as $\gamma$


\texttt{\textbackslash Gamma} is rendered as $\Gamma$


\texttt{\textbackslash gimel} is rendered as $\gimel$


\texttt{\textbackslash hslash} is rendered as $\hslash$


\texttt{\textbackslash imath} is rendered as $\imath$


\texttt{\textbackslash intercal} is rendered as $\intercal$


\texttt{\textbackslash iota} is rendered as $\iota$


\texttt{\textbackslash jmath} is rendered as $\jmath$


\texttt{\textbackslash kappa} is rendered as $\kappa$


\texttt{\textbackslash lambda} is rendered as $\lambda$


\texttt{\textbackslash Lambda} is rendered as $\Lambda$


\texttt{\textbackslash mho} is rendered as $\mho$


\texttt{\textbackslash mu} is rendered as $\mu$


\texttt{\textbackslash natural} is rendered as $\natural$


\texttt{\textbackslash nu} is rendered as $\nu$


\texttt{\textbackslash omega} is rendered as $\omega$


\texttt{\textbackslash Omega} is rendered as $\Omega$


\texttt{\textbackslash P} is rendered as $\P$


\texttt{\textbackslash phi} is rendered as $\phi$


\texttt{\textbackslash Phi} is rendered as $\Phi$


\texttt{\textbackslash pi} is rendered as $\pi$


\texttt{\textbackslash Pi} is rendered as $\Pi$


\texttt{\textbackslash pitchfork} is rendered as $\pitchfork$


\texttt{\textbackslash psi} is rendered as $\psi$


\texttt{\textbackslash Psi} is rendered as $\Psi$


\texttt{\textbackslash rho} is rendered as $\rho$


\texttt{\textbackslash S} is rendered as $\S$


\texttt{\textbackslash sigma} is rendered as $\sigma$


\texttt{\textbackslash Sigma} is rendered as $\Sigma$


\texttt{\textbackslash tau} is rendered as $\tau$


\texttt{\textbackslash theta} is rendered as $\theta$


\texttt{\textbackslash Theta} is rendered as $\Theta$


\texttt{\textbackslash top} is rendered as $\top$


\texttt{\textbackslash varepsilon} is rendered as $\varepsilon$


\texttt{\textbackslash varkappa} is rendered as $\varkappa$


\texttt{\textbackslash varnothing} is rendered as $\varnothing$


\texttt{\textbackslash varphi} is rendered as $\varphi$


\texttt{\textbackslash varpi} is rendered as $\varpi$


\texttt{\textbackslash varrho} is rendered as $\varrho$


\texttt{\textbackslash varsigma} is rendered as $\varsigma$


\texttt{\textbackslash vartheta} is rendered as $\vartheta$


\texttt{\textbackslash wp} is rendered as $\wp$


\texttt{\textbackslash xi} is rendered as $\xi$


\texttt{\textbackslash Xi} is rendered as $\Xi$


\texttt{\textbackslash zeta} is rendered as $\zeta$



\section{Identifier variants}\label{sc.var}
The following variants are supported\footnote{Note that \texttt{\textbackslash mathcal} is not available for lowercase Latin letters.}:

\input{commands}
\end{document}
